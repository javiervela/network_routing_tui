\documentclass{article}
% \documentclass[10pt]{article}

\usepackage[utf8]{inputenc}
% \usepackage[spanish]{babel}
% \usepackage{fontspec}

% Packages for formatting
% \usepackage{indentfirst}

% Packages for math and proofs
\usepackage{amsmath, amssymb, amsthm}

% Packages for code formatting
\usepackage{listings}
\usepackage{color}
\usepackage[table]{xcolor}
\usepackage{booktabs}
\usepackage{chngpage}
\usepackage{amsfonts}
 \usepackage{mathtools}

% Packages for graphics and images
\usepackage{graphicx, eso-pic}
\usepackage{float}

% Packages for hyperlinks and document references
\usepackage{hyperref}

% Packages for colors
\usepackage{xcolor}

% Package for modifying titles
% \usepackage{titling}

% Packages for tables
\usepackage{colortbl}
\usepackage{booktabs}
\usepackage{longtable}

% Package for headers and footers
\usepackage{fancyhdr}

% Package for setting page margins
% \usepackage[a4paper, margin=1in]{geometry}

% Package for customizing lists
\usepackage{enumitem}

% Package for URLs
\usepackage{url}

% Package for page size
\usepackage[letterpaper, margin=1in]{geometry}

% Package for line spacing
\usepackage{setspace}
\setstretch{1.2}

\usepackage{titlesec}

% inhibit page breaks in the middle of paragraphs
\widowpenalties 1 10000
\raggedbottom

% Define colors for code formatting
\definecolor{codebg}{rgb}{0.95,0.95,0.95}
\definecolor{codeborder}{rgb}{0.8,0.8,0.8}

% Listings settings for code formatting
\lstset{
    backgroundcolor=\color{codebg},
    basicstyle=\ttfamily\small,
    breaklines=true,
    frame=single,
    rulecolor=\color{codeborder},
    numbers=left,
    numberstyle=\tiny\color{gray},
    xleftmargin=15pt,
    framexleftmargin=15pt,
    captionpos=b, % Position the caption below the listing
    keywordstyle=\color{blue}, % Color the keywords
    commentstyle=\color{green!60!black}, % Color the comments
    stringstyle=\color{red}, % Color the strings
}


% Custom parameters
\newcommand{\assignmentname}{Final Project: Routing Algorithms}
\newcommand{\subjectname}{CS513: Computer Networks}
\newcommand{\titletext}{\assignmentname}

\newcommand{\studentone}{Automne Petitjean}
\newcommand{\studentoneemail}{mpetitjean@wpi.edu}
\newcommand{\studenttwo}{Javier Vela Tambo}
\newcommand{\studenttwoemail}{jvela@wpi.edu}

\author{\studentone \\ \texttt{\studentoneemail} \and \studenttwo \\ \texttt{\studenttwoemail}}

% \newcommand{\submissiondate}{\today}
\date{\today}

\newcommand{\logo}{logo/wpi.png}
\newcommand{\logoheader}{logo/wpi_header.png}

\title{
    \vspace{0.5cm}
    \titletext \\
    \vspace{0.35cm}
    \large \subjectname
}

% Title formatting
\titleformat{\section}{\normalfont\Large\bfseries}{\thesection}{1em}{}

% Add pictures after the title, in the foreground
\AddToShipoutPictureFG*{
    \AtPageUpperLeft{
        \hspace*{\dimexpr\oddsidemargin + \hoffset + 1.5in\relax}
        \raisebox{\dimexpr-\height-2\baselineskip}{
            \makebox[0pt]{\includegraphics[width=5cm]{\logo}}}}
    % 2nd picture on the right
    % \AtPageUpperLeft{
    %     \hspace*{\dimexpr\oddsidemargin + \textwidth + \hoffset + 1.5in - 5cm\relax}
    %     \raisebox{\dimexpr-\height-2\baselineskip}{
    %         \makebox[0pt]{\includegraphics[width=5cm]{\logo}}}}
}

% Header setup
\pagestyle{fancy}
\fancyhf{}
\lhead{\includegraphics[height=0.5cm]{\logoheader}}
\rhead{\textbf{\subjectname}}
\fancyfoot[C]{\thepage}
\renewcommand{\headrulewidth}{0.4pt}

\setlist[itemize]{partopsep=0pt, parsep=0pt, itemsep=0.2em, left=1.5em}
\setlength{\parskip}{0.5em}


\begin{document}

\thispagestyle{empty}

\maketitle

% TODO check bellow before closing report

% The final report should be a well-presented technical report discussing your project (in pdf format). If your project is primarily a programming effort, you should explain how the program works, give specific sample runs and analyze the results. The report should be 5-10 pages in length, although may be less if you do the standard project. If your project team is more than one person then the report should indicate what each person did on the project.”
% 
% - 5-10 pages in length
% - you should explain how the program works
% - give specific sample runs
% - analyze the results.
% - indicate what each person did on the project.
%
% Again, the final project and report are due Friday, December 13, 2025.
% 
% ## Report Details
% 
% In addition to the overview provided above, your report needs to address each of the following questions. You are encouraged to explicitly use this format in organizing your report.
% 
% - What is your name(s)?
% - What project did you select?
% - Describe the work did you do on the project including any deviations from the description if you chose standard project.
% - What are the results you obtained? As appropriate, it is important you include sample output demonstrating functionality/results of your project.
% - What went well on the project?
% - What did you learn?
% - What was hard?
% - What, if anything, did you do beyond the standard project description (if you based your project on it)?
% - What, if anything, did you not complete that was in the basic project description or your initial design?

\section{Introduction}

% - What project did you select?

For the final project, we selected the standard ``Network Layer Routing'' project. The goal of the project is to develop an application to create different network topologies (represented as graphs), execute the link-state and distance-vector routing algorithms, and query the routing tables for each network node. We implement the basic requirements of the standard project and extend it in the following ways:

% - Describe the work did you do on the project including any deviations from the description if you chose standard project.

% TODO our main contributions are...
% TODO additionally we have ...

\begin{itemize}
    \item A Textual User Interface (TUI) to interact with the application.
    \item Additional features:
          \begin{itemize}
              \item Visualization of the network topology.
              \item Save and load network topologies from files.
              \item Load scripted commands from a file to automate tests.
          \end{itemize}
    \item An analysis of the performance of both algorithms in terms of convergence time and number of messages exchanged. % TODO check Automne
    \item A variation of the distance vector algorithm that includes a "via" field in the routing table to prevent "counting to infinity" issues. % TODO check Automne
\end{itemize}

The rest of the report is organized as follows: Section \ref{sec:implementation} describes the implementation of the project, including the algorithms used. Section \ref{sec:results} presents the experimentation and results obtained. Finally, Section \ref{sec:conclusions} concludes the report with reflections on the project.

\section{Implementation} \label{sec:implementation}

% - What are the results you obtained? As appropriate, it is important you include sample output demonstrating functionality/results of your project.

% - you should explain how the program works

% - give specific sample runs

\subsection{Algorithms}

% TODO maybe explain them one by one?

In this project, we compare two different algorithms: the distance vector algorithm and the link state algorithm.

We implemented both methods and compared them to get an idea of the usefulness of such a failsafe. % TODO

\subsubsection*{Link State}

The link state algorithm is a simple graph traversal that looks at all the possible routes from one specific node.

% TODO more details?

\subsubsection*{Distance Vector (legacy)} % TODO make sure this is legacy explanation

% TODO talk about the algorithm itself

The distance vector algorithm is run step by step. During one step, each node sends its current routing tables to all its neighbours. To simulate this sending process, we save a copy of all the routing tables at time $t$. Then this copy is given to each neighbour and used to build the new routing table at time $t+1$.

The new table for $A$ at $t+1$ is built by creating a table with only one route: the route to $A$ with a distance of 0. It is then updated by the received tables from all neighbours. In our experiment, all tables are sent and received simultaneously. In a context where tables arrive irregularly, we would need to remove only the routes going through $B$ when receiving a table from $B$. We would also need to remove routes going through neighbours when we detect that we're not connected to them anymore.

When receiving a table from a neighbour, a node uses it to update its own table. Let's call $A$ the receiving node and $B$ the sender. $A$ compares each route in the $B$ table (to which the distance from $A$ to $B$ has been added) with the routes it already knows and keep the shortest option.

\subsubsection*{Distance vector (enhanced)}

In the classical distance vector algorithm, that we call legacy in this report, tables sent only contains a destination and a distance for each routes. This can create "counting to infinity" issues. In this work we will compare this legacy version with our own variation that includes destination, distance and via for each route. This allows us to ignore that goes through ourselves. If $B$ thinks it can reach $C$ through $A$, $A$ necessarily knows better and don't need this information. Taking it into account runs the risk of propagating wrong information.

\subsection{Application Interface}

% TODO javier

% README from repo

% TODO app only includes distance vector (enhanced)

% - What are the results you obtained? As appropriate, it is important you include sample output demonstrating functionality/results of your project.
% - you should explain how the program works
% - give specific sample runs

\section{Experimentation and Results} \label{sec:results}

% TODO BLABLABLA

\subsection{Error Metrics}

In this report we measure the errors in routing tables when using the dv algorithm compared to ls. We want to know in how many iterations of dv the routing table reaches an accurate result.

There are several ways of measuring the difference between the ideal routing table from ls with the routing table from dv at step $t$. Let us describe the various metrics used and their pros and cons. But first, here are some useful notations.

Let us note:
\begin{itemize}
    \item $G$ the graph
    \item $A$, $B$, $C$... nodes in graph $G$
    \item $d_{ls}(A,B)$ the distance between A and B as given by the link state algorithm
    \item $d_{t}(A,B)$ with $t \in \mathbb{N}$ the distance between A and B in the routing table of A after $t$ iterations.
\end{itemize}

A naïve way to measure the error would be to do this :

$naiveError_{t} = \sum_{A \in G} \sum_{B \in G} |d_{ls}(A,B) - d_{t}(A,B)|$

This, however, presents some problems. Firstly, it is ill defined for cases where $A$ and $B$ are not connected or when the dv routing table doesn't know how to reach $B$ even though a route exists. We can imagine another version:

$numberErrors_{t} = \sum_{A \in G} \sum_{B \in G} 1 - \delta_{d_{ls}(A,B), d_{t}(A,B)}$ with  $\delta_{a,b}  = \begin{cases*}
        1 & if  $a = b$ \\
        0 & otherwise
    \end{cases*}$

This would count exactly the number of routes for which the dv table has a different distance from the ideal route.

While this is an appropriate measure, it doesn't help in cases where the routing table knows of the correct distance but the route itself doesn't match. For this we have to introduce a new notation:

$msg_t(A,B)$ is the time it takes a message to go from $A$ to $B$ following the routes given by the dv algorithm at step $t$.

Due to the nature of this measure, it can only be computed when all nodes in the graph have a routing table computed by dv. Some messages may never reach their destination, either because they reach a node that doesn't know how to reach said destination or because it gets stuck in an infinite loop between two nodes (for this specific case we implemented a hop limit). It allows us imagine this error metric:

$msgError_t = \sum_{A \in G} \sum_{B \in G} 1 - \delta_{d_{ls}(A,B), msg_{t}(A,B)}$

This metric is still not perfect. It doesn't take into account that some routing tables can have a correct route stored with a wrong distance. We argue that it is a good metric as it is close real world expectations of routing tables. Each error means that a message took longer to reach the destination than the ideal route. We can also measure only the message that failed to reach their destination. Sending a message through a non-efficient route is less of an issue than being unable to reach the destination at all even though a route exists.

\subsection{Performance Comparison of Distance Vector Implementations} % TODO check title makes sense

% TODO dv-l vs. dv-e performance comparison, analysis of count to infinity problem

% TODO change dv -> DV, ideally, DV-e and DV-l
% TODO change ls -> LS

Our proposed dv algorithm aims to fix the problem of the "count to infinity" posed by the legacy dv algorithm. Let us look at a simple example:

\begin{figure}[h!]
    \centering
    \includegraphics[scale=0.7]{../images/infinity.png}
    \caption{A simple example graph named Armand}
\end{figure}

We assume that dv has been run enough time so that the routing tables are correct. If the edge between $A$ and $B$ suddenly becomes 300, the legacy algorithm will take a long time before correcting the routing tables.

\begin{figure}[h!]
    \centering
    \includegraphics[scale=0.7]{../images/infinity_legacy.png}
    \caption{Evolution of the msgError during 500 iterations of the dv legacy algorithm}
\end{figure}

This is a good illustration of counting to infinity. The bad news took a long time to travel, $B$ and $C$ remain convinced that they can reach A by going through each other, meaning that all messages from $B$ or $C$ to $A$ are lost until the count reached 300. Meanwhile, our own dv algorithm doesn't have any error and after two steps, the routing tables are updated with the correct distances. The absence of errors can be explained by the fact that $B$ refused to take a route from $C$ that would go through itself thus avoiding the infinite loop route that loses messages.
However, our algorithm is not immune to "count to infinity" problems, we just have to find another example like the one in fig \ref{fig:beatrice}

\begin{figure}[h!]
    \centering
    \includegraphics[scale=0.7]{../images/devious.png}
    \caption{Another example graph named Béatrice}
    \label{fig:beatrice}
\end{figure}

If we then remove the edge between $C$ and $D$, and make the edge between $A$ and $D$ more expensive, the dv algorithm will count to infinity. $A$, $B$ and $C$ will create a loop-route between themselves and never know that it loops. Our dv algorithm is therefore not perfect. \\
Looking further into this, we can find examples where our enhanced algorithm is slower to converge than the legacy one. The counter examples were found by going through random graphs. We decided to investigate deeper into the benefits of our new algorithm. 

\begin{figure}[h!]
    \centering
    \includegraphics[scale=0.7]{../images/charles.png}
    \caption{If you set the weight of edge AB to 50, legacy will converge in 2 steps, whereas enhanced will converge in 3.}
    \label{fig:charles}
\end{figure}

\begin{table}[h!]
\begin{center}
    \begin{tabular}{|c|c|c|c|}
	\hline
	Number of Nodes & p=0.1 & p=0.5 & p=0.9 \\
	\hline
	4 & 81.33 & 89.51 & 88.2 \\ 
	\hline
	6 & 42.87 & 50.77 &  48.47 \\
	\hline
	8 & 12.64 & 17.89 & 15.75 \\
	\hline
	10 & 4.8 & 2.88 & 3.65 \\
	\hline
	12 & 0.51 & 1.15 & 1.3 \\
	\hline
	14 & 0.27 & 0.31 & 0.59 \\
	\hline
	16 & 0.18 & 0.49 & 0.28 \\
	\hline
	\end{tabular}
	\caption{Average number of steps to converge for legacy compared to enhanced depending on the number of nodes in the graph and the density p. (positive numbers mean legacy take more steps than enhanced)}
\end{center}
\end{table}

We compared both algorithms on graphs randomly generated by two parameters: the number of nodes and the probability p of an edge between two nodes. For each of these graph we chose an edge that will be damaged (its weight will be set to 500). We compare the number of steps it takes for enhanced to converge compared to legacy. According to these preliminary results, the density of the graph doesn't seem to affect the relative perfomance. On the contrary, the number of nodes has a massive influence. Our method is barely even better when the graphs get bigger, averaging 0.26 less steps for graphs with 16 nodes. \\
At this step, we can make the hypothesis that with bigger graphs, the risk of counting to infinity is lower, thus reducing the usefulness of our enhanced algorithm. But we would like confirmation of that. That's why we decided to count the number of time count to infinity happens. In order to check if a count to infinity is happening, we measure the convergence time of the legacy algorithm on the same graph twice. The first time we increase the chose edge weight to 100, the second time we increase it to 150. If the algorithm takes longer in the second case, it is because it needs time to count the extra 50.\\

\begin{table}[h!]
\begin{center}
    \begin{tabular}{|c|c|c|c|}
	\hline
	Number of Nodes & p=0.1 & p=0.5 & p=0.9 \\
	\hline
	4 & 52.8\% & 49.6\% & 45.2\% \\ 
	\hline
	6 & 21.0\% & 20.4\% & 22.6\% \\
	\hline
	8 & 3.6\% & 6.4\% & 4.4\% \\
	\hline
	10 & 0.6\% & 1.0\% & 1.0\% \\
	\hline
	12 & 0.2\% & 0.2\% & 0.0\%  \\
	\hline
	14 & 0.0\% & 0.0\% & 0.2\% \\
	\hline
	16 & 0.0\% & 0.0\% & 0.0\% \\
	\hline
	\end{tabular}
	\caption{Percentage of count to infinity depending on node amount and density}
\end{center}
\end{table}

There seems to be a correlation between the number of count to infinity and the relative efficiency of our algorithm. At least we know it does what it was designed to do which is avoid some counts to infinity.

\subsection{Performance Analysis for Distance Vector against Link State}

% TODO LS ground truth, evaluate DV-e performance

\section{Conclusions} \label{sec:conclusions}

% - What went well on the project?

% - What did you learn?

% - What was hard?

% - What, if anything, did you do beyond the standard project description (if you based your project on it)?

Our project went beyond expectations by providing an interactive interface, our own variation of the distance vector algorithm complete with an analysis of its merits. 

% - What, if anything, did you not complete that was in the basic project description or your initial design?

% - Recap results

\section*{Author Contributions}

Conceptualization: A.P., J.V.T.; Algorithm Implementation: A.P.; Interface Implementation: J.V.T.; Testing: A.P., J.V.T.; Experimentation and Result Analysis: A.P., J.V.T.; Documentation and Report: A.P., J.V.T.

\end{document}